% !TEX TS-program = pdflatex
% !TEX encoding = UTF-8 Unicode
% !TEX spellcheck = en_US
% !BIB TS-program = natbib
% Written by Dylan Gatlin (dgatlin5@gmail.com)
\documentclass[preprint2]{aastex63}

\usepackage{amssymb}        % to get all AMS symbols
\usepackage{graphicx}       % to insert figures
\usepackage{hyperref}       % PDF hyperreferences??
\usepackage{natbib} % For citations [authoryear, sort]
\usepackage[yyyymmdd]{datetime} % For YYYY-MM-DD dates
\usepackage{todonotes}      % In case you want to work on something later
\usepackage[nice]{nicefrac} % For inline fractions
% \usepackage{multicol}       % For twocolumn equations
% \usepackage{multirow}       % For complicated tables
\usepackage{amsmath}        % For calculus notation like partials and derivatives
\usepackage{commath}        % Companion to amsmath
\usepackage{textcomp}
% \usepackage{siunitx}        % Proper unit notation
% \input macros.tex           % For defining common symbols

%%%%%%%%%%%%   All the preamble material:   %%%%%%%%%%%%

\shorttitle{Tidally Locked Thermal Phase Curves}
\shortauthors{Gatlin et al.}

% \degree{Bachelors in Arts}      %  #1 {long descr.}
    % {B.A., Astrophysics}        %  #2 {short descr.}
% \dept{Department of}            %  #1 {designation}
    % {Astrophysical and Planetary Science}       %  #2 {name}

% \advisor{Dr.}               %  #1 {title}
%     {Eric T. Wolf}          %  #2 {name}
% \advisordept{Atmospheric and Oceanic Sciences}
% \reader{Prof. Ann-Marie Madigan}        %  2nd person to sign thesis
% \readerdept{Astrophysical and Planetary Sciences}
% \readerThree{Prof. David Brain}  % 3rd person to sign thesis
% \readerThreedept{Astrophysical and Planetary Sciences}
% \readerFour{Prof. Daniel Jones}
% \readerFourdept{Arts and Sciences Honors Program}

% \defenseDate{\formatdate{3}{5}{2019}} % (m)m (d)d yyyy

%

% \dedication[Dedication]{    % NEVER use \OnePageChapter here.
%     \begin{center}To my Grandpa, Dr. Terry Flanagan\\
%     1938 -- 2018
%     \end{center}
% }
% 
% \acknowledgements{  \OnePageChapter % *MUST* BE ONLY ONE PAGE!
% Thank you to my family who helped encourage me on my way to becoming an
%  astrophysicist. To my dad, my mom, and my sister, Mark, Erin, and Carolyn
%  Gatlin. To my girlfriend, Grace Marshall.

% Thank you to my advisors, Susan (Astromom) Armstrong and David Brain.
%  To the professors I've worked alongside, Prof. Jeremy Darling and
%  Prof. Seth Hornstein. To
%  my research advisors, Prof. Guy Stringfellow and Dr. Eric Wolf.

% Thank you to everyone who read my drafts, Prof. Donald Wilkerson and all my peers
%  in PHYS 3050.

% Thank you to my professors Prof. Zach Berta-Thompson, Prof. Nick Schneider,
%  Prof. John Bally,
%  Dr. Mathis Habich, Prof. Melissa Nigro, Prof. Jen Kay, Prof. Peter Pilewskie,
%  and Prof. Alexandra Jahn. To my high school math teacher, Laurie Buchanan.

% Thank you to my co-investigators, Dr. Eric Wolf, Dr. Ravi Kopparapu,
% Dr. Geronimo Villanueva, and Dr. Avi Mandell.

% \facilities{Exoplanet Archive, Planetary Spectrum Generator}}

% \IRBprotocol{E927F29.001X}    % optional!

% \ToCisShort % use this only for 1-page Table of Contents

% \LoFisShort % use this only for 1-page Table of Figures
% \emptyLoF % use this if there is no List of Figures

% \LoTisShort % use this only for 1-page Table of Tables
% \emptyLoT % use this if there is no List of Tables

%%%%%%%%%%%%%%%%%%%%%%%%%%%%%%%%%%%%%%%%%%%%%%%%%%%%%%%%%%%%%%%%%
%%%%%%%%%%%%%%%       BEGIN DOCUMENT...         %%%%%%%%%%%%%%%%%
%%%%%%%%%%%%%%%%%%%%%%%%%%%%%%%%%%%%%%%%%%%%%%%%%%%%%%%%%%%%%%%%%

\begin{document}
\title{Thermal Phase Curve Simulations on the Terrestrial Exoplanet
 TRAPPIST-1 e}

\author{Dylan Gatlin}
\affiliation{University of Colorado Boulder \\
 2000 Colorado Ave. \\
 Boulder, CO 80309, USA}

\begin{abstract}
In order to better direct future exoplanetary research, we must be
 able to accurately predict what telescopes like the James Webb Space Telescope
 (JWST) will be able to detect. I have constructed a data pipeline that takes
 3D climate models as inputs, then runs a line by line radiative transfer model
 called the Planetary Spectrum Generator (PSG) to generate simulated spectra.
 This pipeline can help determine which features of an exoplanetary atmosphere
 are observable using JWST. This paper focuses primarily on a strong candidate
 for a habitable exoplanet, TRAPPIST-1 e, which is assumed to be tidally locked.
 Recent climate modeling studies indicate that not all tidally locked exoplanets
 are the same. Exoplanets with long years will have large, substellar clouds,
 while exoplanets with short years will have a significant Coriolis force,
 which will result in a smaller substellar cloud and more intense zonal winds.
 The thermal emission of a planet depends strongly on the spatial
 distribution of clouds, which would change from the perspective of a distant
 observer over the exoplanet's year. Observing an exoplanet over its year will
 produce a thermal phase curve, which may be able to detect features like the
 size of the substellar cloud or the presence of a runaway greenhouse effect.
 Although the simulations are done using spectra, observations of
 thermal phase curves are more well suited for photometry, and would likely need
 to be characterized relative to simulated thermal phase curves.

\end{abstract}
 
\input ./short_paper/introduction.tex
% \input ./background/background.tex
% \input ./models/models.tex
% \input ./methods/methods.tex
% \input ./spectra/spectra.tex
% \input ./tpc/thermal_phase_curves.tex
% \input ./conclusion/conclusion.tex

%%%%%%%%%   then the Bibliography, if any   %%%%%%%%%
\bibliographystyle{aasjournal}  % or "siam", or "alpha", etc.
% \nocite{*}      % list all refs in database, cited or not
\bibliography{thesis_citations}     % Bib database in "refs.bib"

%%%%%%%%%   then the Appendices, if any   %%%%%%%%%
% \appendix
% \input appendix.tex

\end{document}

