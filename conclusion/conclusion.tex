\chapter{Conclusion}
\label{conclusion}
Although JWST is the most powerful infrared telescope to date, and its ability
 to probe the mid-infrared greatly exceeds previous telescopes, exoplanetary
 spectroscopy struggles to overcome the noise. Using aggressive binning, these
 issues can be overcome, and the PSG in conjunction with climate models can be
 very useful in determining how to effectively bin the spectra to maximize
 specific signals. However, this technique as it was used here failed to measure
 the abundance of the atmospheric species \chem{CO_2} and \chem{H_2O}. This does
 not imply that such a measurement is impossible, but other analysis techniques
 will be necessary to produce a meaningful measurement. In the wavelength ranges
 used here, transit spectra can be a reasonably reliable measure of surface
 temperature, and therefore habitability, and in 10 transits or less, the noise
 can be reasonably controlled for this purpose.

Transit spectroscopy suffers from some inherent limitations, such as it can only
 probe the terminator profile, and is therefore most useful for well-mixed
 atmospheric species only. Other atmospheric parameters like clouds are
 difficult to see in spectra, and it is not the most useful technique for
 those objectives.

If we wish to completely understand the behavior of a planet, we must use as
 many unique observational techniques as possible because each will provide
 different insights. In this effort, thermal phase curves are a promising
 technique, and they are the only method capable of analyzing an exoplanet's
 cloud distribution. Thermal phase curves also produce dramatically different
 signals depending on whether the planet is a fast rotator, a slow rotator, or too
 hot to sustain life. However, more studies need to be done on different
 cases to fully understand the relationship between thermal phase curve
 shape and the atmosphere being probed. 3D climate models and the PSG are
 two useful tools, especially when used together.

The optimal wavelengths for thermal phase curves are the longest wavelengths,
 particularly past $17\um$, which is well suited to the F2550W and F2100W filters
 on JWST. Even the most optimistic estimates of MIRI's spectral resolution show
 that thermal phase curves are not not well suited for a spectrograph as the
 signal to noise ratios are far too low. Unfortunately, the PSG is not designed
 to estimate signal to noise of MIRI's imaging instruments, so other tools
 would need to be used to estimate the signal to noise in those cases.

Thermal phase curves will be far more complicated on exoplanetary systems
 with more than one planet, which is the case for TRAPPIST-1. Although the
 results here show promise for the technique of thermal phase curves, the process
 of implementing this technique in a observation would involve computing thermal
 phase curves for multiple planets at once. This can be overcome with advanced
 reduction algorithms, and the PSG with 3D climate models would likely serve as
 a useful tool for matching the data to an atmosphere. Thermal phase curves
 would be most fruitful in single-planet systems near the habitable zone, as
 they were implemented here.

As exoplanetary astrophysics evolves over the next few years, the need for
 accurate climate models will only increase. Climate models are the only way we
 can effectively understand and predict an exoplanet's atmosphere, and so far,
 tidally locked exoplanet atmospheres have been dramatically different from
 what we would have expected from Earth's atmosphere. The climate models used
 here are extremely probative, but they are missing many features that will likely
 be important on an exoplanet. The presence of atmospheric species like
 \chem{O_3}, \chem{CH_4}, and others are important additions to future models,
 as well as the integration of M-dwarf solar physics like stellar flares, which
 will likely be a major source of atmospheric loss. Carefully integrating these
 features into existing climate models will greatly improve the probative value
 of climate models, particularly when used with the PSG to integrate them into
 spectra simulations. The probative potential of climate models in conjunction
 with the PSG goes far beyond the results discussed in this paper, and together
 they serve as a great technique for furthering our understanding of potentially
 habitable worlds.