\chapter{Introduction}
% \section{Introduction}
\label{introchap}
Exoplanetary science is one of the most exciting fields in astrophysics. Until
 the Kepler mission, it was unknown how
 common exoplanets may be, but we have since estimated that approximately one
 third of all F, G, and K stars have at least one terrestrial exoplanet
 \citep{nexoplanets}. Excitingly, we have found a number of terrestrial
 exoplanets which resemble Earth in mass, size, and most crucially, solar
 irradiance. These terrestrial exoplanets are the most likely candidates in the
 search for habitable worlds beyond Earth, and by extension, the most likely
 places to discover extraterrestrial life or a future home for humanity. To
 date, the list of exoplanets in the classical habitable zone is short, and few
 are close to Earth, but this will soon change with the recently launched
 Transiting Exoplanet Survey Satellite (TESS). TESS is expected to discover
 thousands of exoplanets smaller than Neptune and dozens of Earth sized planets,
 and was optimized for finding planets that are closer to Earth
 \citep{tesspredict}. In conjunction with the James Webb Space Telescope (JWST),
 we expect to dramatically improve observations of potentially habitable
 worlds in nearby star systems.

The most easily detectable exoplanets orbit M-dwarfs, which are very low mass,
 dim stars. These exoplanets are likely tidally locked due to their close orbit
 to their host star \citep{dynamicsfate}, meaning they don't rotate relative to
 their star. The solar system contains no example of such a planet; however, the
 Earth-Moon system does provide an example as the Moon is tidally locked to
 Earth. The
 atmospheres of tidally locked planets behave differently from any atmosphere
 we are previously
 familiar with, so the only way to characterize their atmospheres is via global
 circulation models (GCMs). GCMs were invented to predict
 weather or climate change, and have proven scientifically valuable for a number
 of different
 planets throughout the solar system \citep{venusgcm}. Many astronomers have
 made predictions of what JWST will see on exoplanets, but most of these
 predictions assume that habitable exoplanet atmospheres will be replicas of
 Earth \citep{ostproposal}. However, recent 3D climate modeling experiments
 demonstrate that exoplanets are typically very different than Earth. For
 example, an
 exoplanet that receives approximately the same amount of sunlight as Earth,
 TRAPPIST-1 d, is less likely to be habitable than TRAPPIST-1 e, a planet that
 receives only 60\% of the solar irradiance of Earth \citep{wolf18}.

With the launch of JWST on the horizon, it is essential that we know what to
 expect from observations, and atmospheric models are the best method
 for constraining predictions. Simply assuming a tidally locked
 exoplanet will appear identical to Earth is unreasonable and can lead to
 inaccurate conclusions. Using GCM simulations for planets in the TRAPPIST-1
 system, we can self-consistently predict what the TRAPPIST-1 exoplanets may look
 like with regards to cloud distributions, precipitation, and temperature.
 NASA's Planetary Spectrum Generator (PSG) can then be used to generate spectra
 for these climate models, giving us more accurate predictions of what JWST
 might be
 able to see \citep{psgpaper}. Using the PSG, we can determine which spectral 
 features would be detectable with JWST and using signal to noise analysis, we
 will help the astrophysics community prioritize their time with JWST.

In addition to exoplanet transit spectra, GCMs allow us to study exoplanets
 using other
 methods, most notably thermal emission phase curves. A GCM provides global
 resolution of an exoplanet's surface and atmosphere, but only a small fraction
 of it can be probed via the transit method. The PSG can allow us to measure the
 thermal emission of the planet as it rotates relative to the Earth over its
 year. Thermal phase curves measure the change in thermal brightness of an
 exoplanet as it orbits its star, and can be variable over time. Thermal phase
 curves have the advantage of providing unique signals depending on the climate
 of the planet. This technique
 serves as the best method to resolve features like clouds on exoplanets, and
 by extension, we may infer details about the planet's climate. Thermal phase
 curves and transit spectra provide two very unique methods of probing an
 exoplanet's atmosphere, and together, they will help JWST find habitable worlds
 beyond Earth.

In this thesis, I will calculate both transit spectra and thermal phase curves
 of several habitable zone exoplanets, particularly TRAPPIST-1 e. In the
 following section, Chapter \ref{background}, I will explain the fundamentals of
 exoplanet research and exoplanet observations, with an emphasis on the
 variables significant to thermal phase curves and transits. In Chapter
 \ref{models}, I will review climate models from \citet{wolf17, wolf18}, and
 particularly focus on the terminator atmosphere profile and the global cloud
 patterns. I will compare fast rotators and slow rotators, which have
 dramatically different cloud patterns, and therefore different thermal phase
 curves. In Chapter \ref{methods}, I will explain the data pipeline around the
 PSG and the
 PSG as a tool to simulate spectra. In Chapter \ref{spectra}, I will show
 transit spectra results, analyze their behavior, and identify and measure
 prominent features. In Chapter \ref{thermalphasecurves}, I will do the
 equivalent analysis for thermal phase curves, and compare slow rotators thermal
 phase curves to fast rotators.
 In Chapter \ref{conclusion}, I will compare the transit spectra and thermal
 phase curves, identifying their respective strengths and weaknesses.
