\section{Introduction}
\label{sec:intro}
We have found a number of terrestrial
 exoplanets which resemble Earth in mass, size, and most crucially, solar
 irradiance. These terrestrial exoplanets are the most likely candidates in the
 search for habitable worlds beyond Earth, and by extension, the most likely
 places to discover extraterrestrial life or a future home for humanity. To
 date, the list of exoplanets in the classical habitable zone is short, and few
 are close to Earth, but this will soon change with the recently launched
 Transiting Exoplanet Survey Satellite (TESS). TESS is expected to discover
 thousands of exoplanets smaller than Neptune and dozens of Earth sized planets,
 and was optimized for finding planets that are closer to Earth
 \citep{tesspredict}. In conjunction with the James Webb Space Telescope (JWST),
 we can expect increased quality and quantity of habitable exoplanet
 observations.

The most easily detectable exoplanets orbit M-dwarfs. These exoplanets are
 likely tidally locked due to their close orbit
 to their host star \citep{dynamicsfate}.The
 atmospheres of tidally locked planets behave differently from any atmosphere
 we are previously
 familiar with, so the only way to characterize their atmospheres is via global
 circulation models (GCMs). GCMs were invented to predict
 weather or climate change, and have proven scientifically valuable for a number
 of different
 planets throughout the solar system \citep{venusgcm}. Many astronomers have
 made predictions of what JWST will see on exoplanets, but many of these
 predictions assume that habitable exoplanet atmospheres will be replicas of
 Earth \citep{ostproposal}. However, recent 3D climate modeling experiments
 demonstrate that exoplanets are typically very different than Earth. For
 example, \citet{wolf18} found that although TRAPPIST-1 d receives approximately
 the same amount of solar irradiance as Earth, TRAPPIST-1 e makes a better
 candidate for habitability with only 60\% the solar irradiance of Earth.

With the launch of JWST on the horizon, it is essential that we know what to
 expect from observations, and atmospheric models are the best method
 for constraining predictions. Simply assuming a tidally locked
 exoplanet will appear identical to Earth is unreasonable and can lead to
 inaccurate conclusions. Using GCM simulations for planets in the TRAPPIST-1
 system, we can self-consistently predict what the TRAPPIST-1 exoplanets may look
 like with regards to cloud distributions, precipitation, and temperature.
 NASA's Planetary Spectrum Generator (PSG) can then be used to generate spectra
 for these climate models, giving us more accurate predictions of what JWST
 might be able to see \citep{psgpaper}.

The PSG can allow us to measure the
 thermal emission of the planet as it rotates relative to the Earth over its
 year. Thermal phase curves measure the change in thermal brightness of an
 exoplanet as it orbits its star, and can be variable over time. This technique
 serves as the best method to resolve features like clouds on exoplanets, and
 by extension, the planet's climate.

Here I will create thermal phase curves
 of several habitable zone exoplanets, but first I will review climate models
 from \citet{wolf17, wolf18}, focusing on characterizing the surface temperature
 and the substellar
 clouds of those models, as well as comparing fast and slow rotators. I
 will explain the data pipeline around the PSG, which I will use to simulate
 spectra that will be reduced into thermal phase curves.
